\documentclass[12pt,a4paper,norsk]{article}
\usepackage[utf8]{inputenc}

\setlength\parindent{0pt}
\setlength{\parskip}{1\baselineskip}

\begin{document}

\author{Martin S{\o}yland\footnote{E-mail: martin.g.soyland@gmail.com} og
        Solveig Bj{\o}rkholt\footnote{E-mail: solveig.bjorkholt@stv.uio.no}}
\title{\LARGE{STV2XXX \\ Store tekstdata}: \\ \large{Samling, strukturering og analyse av politiske tekster}}
\maketitle

\section*{Kort om emnet}
Hvordan samler man effektivt og redelig store mengder politiske tekster? Hva må til for å gjøre slike tekster klare for analyse? Og hvordan kan vi analysere tekstene?

Politikere og politiske partier produserer store mengder tekst hver dag. Om det er gjennom debatter, taler på Stortinget, lovforslag fra regjeringen, høringer, offentlige utredninger med mer, er digitaliserte politiske tekster i det offentlige blitt mer tilgjengelig de siste tiårene. Dette har åpnet et mulighetsrom for tekstanalyse som ikke var mulig/veldig vanskelig og tidkrevende før.

Det kan ofte være vanskelig å finne mønster som kan svare på spørsmål og teorier vi har i statsvitenskap i disse store tekstsamlingene. Derfor kan vi se til metoder innenfor maskinlæring for å analysere store samlinger av tekst systematisk. Samtidig er ikke alltid digitaliserte politiske tekster er tilrettelagt for å analysers direkte. I disse tilfellene er god strukturering av rådata viktig.

I dette kurset vil du lære å søke i store mengder dokumenter, oppsummere disse på meningsfulle måter og indentifisere riktige analysemetoder for å teste statsvitenskaplige teorier. Kurset vil dekke samling av store volum tekst fra offentlige kilder, strukturering og klargjøring av tekst for analyse og kvantitative tekstanalysemetoder.

\section*{Hva lærer du?}

\subsection*{Kunnnskaper}

Du vil lære:
\begin{itemize}
  \item teknikker for effektiv samling av store mengder tekst fra offentlige kilder
  \item strukturere og klargjøre dokumentsamlinger for kvanitativ analyse
  \item bruksområder for tekst i statsvitenskaplig analyse
  \item analysemetoder for å trekke ut meningsfulle innsikter fra dokumentsamlinger
\end{itemize}

\subsection*{Ferdigheter}

Du vil kunne:
\begin{itemize}
  \item samle store mengder data fra strukturerte og ustrukturerte offentlige kilder
  \item klargjøre store dokumentsamlinger for analyse
  \item identifisere passende analyseteknikk for å besvare en problemstilling
  \item oppsummere og presentere analyseresultater av tekst på en forståelig måte
\end{itemize}


\subsection*{Generell kompetanse}

Du vil kunne:

\begin{itemize}
  \item gjøre utvalg og tilpasse kvantitative tekstanalyseteknikker for eget arbeid
  \item samle, strukturere, preprosessere og endre store mengder tekst
  \item anvende forskjellige analysemetoder og oppsummere resultater på en meningsfull måte
  \item utføre arbeidslivsrelevant tekstbehandling og analyse
\end{itemize}


\section*{Undervisning}

Forelesning og seminar.



\subsection*{Forelesninger}

Forelesningene vil dekke hovedtemaene i kurset med eksempler fra forskning og arbeidsliv. Tema for forelesning og seminar vil bli delt opp i hvordan å:

\begin{itemize}
  \item anskaffe politiske tekster i forskjellige format
  \item strukturere tekster i form av kvantitative data
  \item oppsummere mengder tekst
  \item finne riktig analysemetode til tekstsamlinger for å svare på konkrete problemstillinger
\end{itemize}

Forslag til forelesninger:

\begin{enumerate}
  \item Introduksjon
  \item Anskaffelse og innlasting av tekst (a)
  \item Bruke API -- case: Stortinget
  \item Forbehandling av tekst (a)
  \item Forbehandling av tekst (b)
  \item Veiledet læring (supervised learning) vs. ikke-veiledet læring (unsupervised learning)
  \item Ordbøker og likhet mellom tekster
  \item Klassifisering av tekst -- temamodellering
  \item Estimerer latent (ideologisk) posisjon fra tekst
  \item Oppsummering
\end{enumerate}

\subsection*{Seminarer}

I seminarene vil studentene lære å utføre datasamling, strukturering av data og analyse av data, knyttet til forelesningene.

Seminarene vil fungere som et springbrett for deloppgavene over semesteret, der man går gjennom generelle trekk ved den aktuelle oppgaven.

\section*{Obligatorisike aktiviteter}

\begin{itemize}
  \item Oppmøte på seminarene
  \item Levere oppgave etter hvert seminar
\end{itemize}


\section*{Eksamen}

Semesteroppgave.

Semesteroppgaven bygges gradvis i løpet av seminarrekka. Delene samles til en fullstendig oppgave og leveres på slutten av semesteret.

\end{document}
