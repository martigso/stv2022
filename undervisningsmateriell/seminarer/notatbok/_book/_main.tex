% Options for packages loaded elsewhere
\PassOptionsToPackage{unicode}{hyperref}
\PassOptionsToPackage{hyphens}{url}
%
\documentclass[
]{book}
\usepackage{amsmath,amssymb}
\usepackage{lmodern}
\usepackage{iftex}
\ifPDFTeX
  \usepackage[T1]{fontenc}
  \usepackage[utf8]{inputenc}
  \usepackage{textcomp} % provide euro and other symbols
\else % if luatex or xetex
  \usepackage{unicode-math}
  \defaultfontfeatures{Scale=MatchLowercase}
  \defaultfontfeatures[\rmfamily]{Ligatures=TeX,Scale=1}
\fi
% Use upquote if available, for straight quotes in verbatim environments
\IfFileExists{upquote.sty}{\usepackage{upquote}}{}
\IfFileExists{microtype.sty}{% use microtype if available
  \usepackage[]{microtype}
  \UseMicrotypeSet[protrusion]{basicmath} % disable protrusion for tt fonts
}{}
\makeatletter
\@ifundefined{KOMAClassName}{% if non-KOMA class
  \IfFileExists{parskip.sty}{%
    \usepackage{parskip}
  }{% else
    \setlength{\parindent}{0pt}
    \setlength{\parskip}{6pt plus 2pt minus 1pt}}
}{% if KOMA class
  \KOMAoptions{parskip=half}}
\makeatother
\usepackage{xcolor}
\IfFileExists{xurl.sty}{\usepackage{xurl}}{} % add URL line breaks if available
\IfFileExists{bookmark.sty}{\usepackage{bookmark}}{\usepackage{hyperref}}
\hypersetup{
  pdftitle={STV2022 -- Store tekstdata},
  pdfauthor={Solveig Bjørkholt og Martin Søyland},
  hidelinks,
  pdfcreator={LaTeX via pandoc}}
\urlstyle{same} % disable monospaced font for URLs
\usepackage{color}
\usepackage{fancyvrb}
\newcommand{\VerbBar}{|}
\newcommand{\VERB}{\Verb[commandchars=\\\{\}]}
\DefineVerbatimEnvironment{Highlighting}{Verbatim}{commandchars=\\\{\}}
% Add ',fontsize=\small' for more characters per line
\usepackage{framed}
\definecolor{shadecolor}{RGB}{248,248,248}
\newenvironment{Shaded}{\begin{snugshade}}{\end{snugshade}}
\newcommand{\AlertTok}[1]{\textcolor[rgb]{0.94,0.16,0.16}{#1}}
\newcommand{\AnnotationTok}[1]{\textcolor[rgb]{0.56,0.35,0.01}{\textbf{\textit{#1}}}}
\newcommand{\AttributeTok}[1]{\textcolor[rgb]{0.77,0.63,0.00}{#1}}
\newcommand{\BaseNTok}[1]{\textcolor[rgb]{0.00,0.00,0.81}{#1}}
\newcommand{\BuiltInTok}[1]{#1}
\newcommand{\CharTok}[1]{\textcolor[rgb]{0.31,0.60,0.02}{#1}}
\newcommand{\CommentTok}[1]{\textcolor[rgb]{0.56,0.35,0.01}{\textit{#1}}}
\newcommand{\CommentVarTok}[1]{\textcolor[rgb]{0.56,0.35,0.01}{\textbf{\textit{#1}}}}
\newcommand{\ConstantTok}[1]{\textcolor[rgb]{0.00,0.00,0.00}{#1}}
\newcommand{\ControlFlowTok}[1]{\textcolor[rgb]{0.13,0.29,0.53}{\textbf{#1}}}
\newcommand{\DataTypeTok}[1]{\textcolor[rgb]{0.13,0.29,0.53}{#1}}
\newcommand{\DecValTok}[1]{\textcolor[rgb]{0.00,0.00,0.81}{#1}}
\newcommand{\DocumentationTok}[1]{\textcolor[rgb]{0.56,0.35,0.01}{\textbf{\textit{#1}}}}
\newcommand{\ErrorTok}[1]{\textcolor[rgb]{0.64,0.00,0.00}{\textbf{#1}}}
\newcommand{\ExtensionTok}[1]{#1}
\newcommand{\FloatTok}[1]{\textcolor[rgb]{0.00,0.00,0.81}{#1}}
\newcommand{\FunctionTok}[1]{\textcolor[rgb]{0.00,0.00,0.00}{#1}}
\newcommand{\ImportTok}[1]{#1}
\newcommand{\InformationTok}[1]{\textcolor[rgb]{0.56,0.35,0.01}{\textbf{\textit{#1}}}}
\newcommand{\KeywordTok}[1]{\textcolor[rgb]{0.13,0.29,0.53}{\textbf{#1}}}
\newcommand{\NormalTok}[1]{#1}
\newcommand{\OperatorTok}[1]{\textcolor[rgb]{0.81,0.36,0.00}{\textbf{#1}}}
\newcommand{\OtherTok}[1]{\textcolor[rgb]{0.56,0.35,0.01}{#1}}
\newcommand{\PreprocessorTok}[1]{\textcolor[rgb]{0.56,0.35,0.01}{\textit{#1}}}
\newcommand{\RegionMarkerTok}[1]{#1}
\newcommand{\SpecialCharTok}[1]{\textcolor[rgb]{0.00,0.00,0.00}{#1}}
\newcommand{\SpecialStringTok}[1]{\textcolor[rgb]{0.31,0.60,0.02}{#1}}
\newcommand{\StringTok}[1]{\textcolor[rgb]{0.31,0.60,0.02}{#1}}
\newcommand{\VariableTok}[1]{\textcolor[rgb]{0.00,0.00,0.00}{#1}}
\newcommand{\VerbatimStringTok}[1]{\textcolor[rgb]{0.31,0.60,0.02}{#1}}
\newcommand{\WarningTok}[1]{\textcolor[rgb]{0.56,0.35,0.01}{\textbf{\textit{#1}}}}
\usepackage{longtable,booktabs,array}
\usepackage{calc} % for calculating minipage widths
% Correct order of tables after \paragraph or \subparagraph
\usepackage{etoolbox}
\makeatletter
\patchcmd\longtable{\par}{\if@noskipsec\mbox{}\fi\par}{}{}
\makeatother
% Allow footnotes in longtable head/foot
\IfFileExists{footnotehyper.sty}{\usepackage{footnotehyper}}{\usepackage{footnote}}
\makesavenoteenv{longtable}
\usepackage{graphicx}
\makeatletter
\def\maxwidth{\ifdim\Gin@nat@width>\linewidth\linewidth\else\Gin@nat@width\fi}
\def\maxheight{\ifdim\Gin@nat@height>\textheight\textheight\else\Gin@nat@height\fi}
\makeatother
% Scale images if necessary, so that they will not overflow the page
% margins by default, and it is still possible to overwrite the defaults
% using explicit options in \includegraphics[width, height, ...]{}
\setkeys{Gin}{width=\maxwidth,height=\maxheight,keepaspectratio}
% Set default figure placement to htbp
\makeatletter
\def\fps@figure{htbp}
\makeatother
\setlength{\emergencystretch}{3em} % prevent overfull lines
\providecommand{\tightlist}{%
  \setlength{\itemsep}{0pt}\setlength{\parskip}{0pt}}
\setcounter{secnumdepth}{5}
\usepackage{booktabs}
\ifLuaTeX
  \usepackage{selnolig}  % disable illegal ligatures
\fi
\usepackage[]{natbib}
\bibliographystyle{plainnat}

\title{STV2022 -- Store tekstdata}
\author{Solveig Bjørkholt og Martin Søyland}
\date{2022-06-01}

\begin{document}
\maketitle

{
\setcounter{tocdepth}{1}
\tableofcontents
}
\hypertarget{introduksjon}{%
\chapter{Introduksjon}\label{introduksjon}}

Velkommen til STV2022 -- Store teksdata!

Hvordan samler man effektivt og redelig store mengder politiske tekster? Hva må til for å gjøre slike tekster klare for analyse? Og hvordan kan vi analysere tekstene?

Politikere og politiske partier produserer store mengder tekst hver dag. Om det er gjennom debatter, taler på Stortinget, lovforslag fra regjeringen, høringer, offentlige utredninger med mer, er digitaliserte politiske tekster i det offentlige blitt mer tilgjengelig de siste tiårene. Dette har åpnet et mulighetsrom for tekstanalyse som ikke var mulig/veldig vanskelig og tidkrevende før.

Det kan ofte være vanskelig å finne mønster som kan svare på spørsmål og teorier vi har i statsvitenskap i disse store tekstsamlingene. Derfor kan vi se til metoder innenfor maskinlæring for å analysere store samlinger av tekst systematisk. Samtidig er ikke alltid digitaliserte politiske tekster er tilrettelagt for å analysers direkte. I disse tilfellene er god strukturering av rådata viktig.

I dette kurset vil du lære å søke i store mengder dokumenter, oppsummere disse på meningsfulle måter og indentifisere riktige analysemetoder for å teste statsvitenskaplige teorier. Kurset vil dekke samling av store volum tekst fra offentlige kilder, strukturering og klargjøring av tekst for analyse og kvantitative tekstanalysemetoder.

\hypertarget{undervisning}{%
\chapter{Undervisning}\label{undervisning}}

Under beskrives undervisningen i STV2022.

\hypertarget{forelesninger}{%
\section{Forelesninger}\label{forelesninger}}

Kurset har 10 forelesninger:

\begin{enumerate}
\def\labelenumi{\arabic{enumi}.}
\tightlist
\item
  \protect\hyperlink{introduksjon}{Intro!} (uke 34)
\item
  \protect\hyperlink{anskaff}{Anskaffelse} og \protect\hyperlink{lastetekst}{innlasting av tekst} (uke 35)
\item
  \protect\hyperlink{prepros}{Forbehandling av tekst 1} (uke 36)
\item
  \protect\hyperlink{prepros}{Forbehandling av tekst 2} (uke 37)
\item
  \protect\hyperlink{anskaff}{Bruke API} (Stortinget) (uke 38)
\item
  \protect\hyperlink{sup}{Veiledet} versus \protect\hyperlink{unsup}{ikke-veiledet} læring (uke 41)
\item
  \protect\hyperlink{ordboker}{Ordbøker}, \protect\hyperlink{tekststats}{tekstlikhet} og \protect\hyperlink{sentiment}{sentiment} (uke 42)
\item
  Klassifisering av tekst -- \protect\hyperlink{topicmod}{temamodellering} (uke 43)
\item
  \protect\hyperlink{posisjon}{Estimere latent posisjon fra tekst} (uke 44)
\item
  \href{oppsummering}{Oppsummering}! (uke 46)
\end{enumerate}

Det er sterkt anbefalt at man ser over pensum før forelesning og seminar.

\hypertarget{pensum}{%
\section{Pensum}\label{pensum}}

\hypertarget{generelt}{%
\subsection{Generelt}\label{generelt}}

Under er en liste over bidrag som vil bli dekket kontinuerlig gjennom hele kurset:

\begin{enumerate}
\def\labelenumi{\arabic{enumi}.}
\tightlist
\item
  \citet{Silge2017}
\item
  \citet{Grimmer2022}
\item
  \citet{Benoit2017}
\item
  \citet{Jurafsky2021b} (anbefalt)
\item
  \citet{Wickham2016} (anbefalt)
\end{enumerate}

\hypertarget{uke-34.-introduksjon-solveig-og-martin}{%
\subsection{Uke 34. Introduksjon (Solveig og Martin)}\label{uke-34.-introduksjon-solveig-og-martin}}

Den første uken vil vi fokusere på de genrelle konseptene innenfor kvantitativ tekstanalyse, prosessen med å gå fra rå tekst til slutningsanalyse, potensielle fallgruver, med mer.

\begin{enumerate}
\def\labelenumi{\arabic{enumi}.}
\tightlist
\item
  \citet{Grimmer2022} kap. 1-2 og 22 (36 sider)
\item
  \citet{Lucas2015} (24 sider)
\item
  \citet{Silge2017} kap. 1 (9 sider)
\item
  \citet{Pang2008} kap. 1 (10 sider)
\end{enumerate}

\hypertarget{uke-35.-anskaffelse-og-innlasting-av-tekst-martin}{%
\subsection{Uke 35. Anskaffelse og innlasting av tekst (Martin)}\label{uke-35.-anskaffelse-og-innlasting-av-tekst-martin}}

\begin{enumerate}
\def\labelenumi{\arabic{enumi}.}
\tightlist
\item
  \citet{Grimmer2022} kap. 3-4 (14 sider)
\item
  \citet{Cooksey2014} kap. 1 (4 sider)
\item
  \citet{Wickham2020} (8 sider)
\item
  \citet{Hoyland2019} (22 sider)
\end{enumerate}

\hypertarget{uke-36.-forbehandling-av-tekst-1-martin}{%
\subsection{Uke 36. Forbehandling av tekst 1 (Martin)}\label{uke-36.-forbehandling-av-tekst-1-martin}}

\begin{enumerate}
\def\labelenumi{\arabic{enumi}.}
\tightlist
\item
  \citet{Grimmer2022} kap. 5 (11 sider)
\item
  \citet{Silge2017} kap. 3 (9 sider)
\item
  \citet{Joergensen2019} (10 sider)
\item
  \citet{Barnes2019} (12 sider)
\item
  \citet{Benoit2020} (11 sider)
\end{enumerate}

\textbf{Seminar 1: Anskaffe tekst og lage dtm i R}

\hypertarget{uke-37.-forbehandling-av-tekst-2-solveig}{%
\subsection{Uke 37. Forbehandling av tekst 2 (Solveig)}\label{uke-37.-forbehandling-av-tekst-2-solveig}}

\begin{enumerate}
\def\labelenumi{\arabic{enumi}.}
\tightlist
\item
  \citet{Grimmer2022} kap. 9 (7 sider)
\item
  \citet{Silge2017} kap. 4 (15 sider)
\item
  \citet{Denny2018} (21 sider)
\end{enumerate}

\hypertarget{uke-38.-bruke-api-case-stortinget-martin}{%
\subsection{Uke 38. Bruke API -- Case: Stortinget (Martin)}\label{uke-38.-bruke-api-case-stortinget-martin}}

\begin{enumerate}
\def\labelenumi{\arabic{enumi}.}
\tightlist
\item
  \citet{datastortinget2022} (1 sider)
\item
  \citet{Soeyland2022} (1 sider)
\item
  \citet{Finseraas2021} (10 sider)
\end{enumerate}

\textbf{Seminar 2: Preprosessering av tekstdata i R}

\hypertarget{uke-39.-ingen-undervisning}{%
\subsection{Uke 39. INGEN UNDERVISNING}\label{uke-39.-ingen-undervisning}}

\hypertarget{uke-40.-ingen-undervisning}{%
\subsection{Uke 40. INGEN UNDERVISNING}\label{uke-40.-ingen-undervisning}}

\hypertarget{uke-41.-veiledet-luxe6ring-og-ikke-veiledet-luxe6ring-solveig}{%
\subsection{Uke 41. Veiledet læring og ikke-veiledet læring (Solveig)}\label{uke-41.-veiledet-luxe6ring-og-ikke-veiledet-luxe6ring-solveig}}

\begin{enumerate}
\def\labelenumi{\arabic{enumi}.}
\tightlist
\item
  \citet{Grimmer2022} kap. 10 og 17 (24 sider)
\item
  \citet{dorazio_separating_2014} (18 sider)
\item
  \citet{feldman_sanger_20061} (17 sider)
\item
  \citet{feldman_sanger_20062} (11 sider)
\item
  \citet{muchlinski_siroky_he_kocher_2016} (16 sider)
\end{enumerate}

\hypertarget{uke-42.-ordbuxf8ker-tekstlikhet-og-sentiment-solveig}{%
\subsection{Uke 42. Ordbøker, tekstlikhet og sentiment (Solveig)}\label{uke-42.-ordbuxf8ker-tekstlikhet-og-sentiment-solveig}}

\begin{enumerate}
\def\labelenumi{\arabic{enumi}.}
\tightlist
\item
  \citet{Grimmer2022} kap. 7 og 16 (12 sider)
\item
  \citet{Silge2017} kap. 2 (11 sider)
\item
  \citet{Pang2008} kap. 3-4 (26 sider)
\item
  \citet{liu_introduction_2015} (15 sider)
\item
  \citet{liu_problem_2015} (30 sider)
\end{enumerate}

\textbf{Seminar 3: Sup vs.~unsup i R}

\hypertarget{uke-43.-klassifisering-av-tekst-temamodellering-martin}{%
\subsection{Uke 43. Klassifisering av tekst -- Temamodellering (Martin)}\label{uke-43.-klassifisering-av-tekst-temamodellering-martin}}

\begin{enumerate}
\def\labelenumi{\arabic{enumi}.}
\tightlist
\item
  \citet{Grimmer2022} kap. 13 og (13 sider)
\item
  \citet{Blei2012} (8 sider)
\item
  \citet{Silge2017} kap. 6 (14 sider)
\item
  \citet{Roberts2014} (19 sider)
\end{enumerate}

\hypertarget{uke-44.-estimere-latent-posisjon-fra-tekst-solveig}{%
\subsection{Uke 44. Estimere latent posisjon fra tekst (Solveig)}\label{uke-44.-estimere-latent-posisjon-fra-tekst-solveig}}

\begin{enumerate}
\def\labelenumi{\arabic{enumi}.}
\tightlist
\item
  \citet{Laver2003} (20 sider)
\item
  \citet{Slapin2008} (18 sider)
\item
  \citet{Lowe2017} (15 sider)
\item
  \citet{Lauderdale2016} (20 sider)
\item
  \citet{Peterson2018} (8 sider)
\end{enumerate}

\textbf{Seminar 4: Modelleringsmetoder i R}

\hypertarget{uke-46.-oppsummering-solveig-og-martin}{%
\subsection{Uke 46. Oppsummering (Solveig og Martin)}\label{uke-46.-oppsummering-solveig-og-martin}}

\begin{enumerate}
\def\labelenumi{\arabic{enumi}.}
\tightlist
\item
  \citet{Grimmer2022} kap 28 (5 sider)
\item
  \citet{Wilkerson2017} (19 sider)
\end{enumerate}

\textbf{Seminar 5: Fra tekst til funn -- Q\&A/Oppg.hjelp}

\hypertarget{seminarer}{%
\section{Seminarer}\label{seminarer}}

Seminarene vil bli

\hypertarget{nyttige-ressurser}{%
\section{Nyttige ressurser}\label{nyttige-ressurser}}

\begin{itemize}
\tightlist
\item
  \href{https://shinyibv02.uio.no/connect/\#/apps/55/access}{Arbeidsbøker for R ved UiO}
\item
  \href{https://github.com/liserodland/STV1020}{R materiale for STV1020}
\end{itemize}

\hypertarget{anskaff}{%
\chapter{Anskaffelse av tekst}\label{anskaff}}

\hypertarget{lastetekst}{%
\chapter{Laste inn tekst}\label{lastetekst}}

\hypertarget{prepros}{%
\chapter{Preprosessering}\label{prepros}}

Preprosessering er ganske viktig.

\hypertarget{sekk-med-ord}{%
\section{Sekk med ord}\label{sekk-med-ord}}

\emph{Alle} tekster er unike!

Ta for eksempel spor 6 på No.4-albumet vi allerede har jobbet med -- \emph{Regndans i skinnjakke}. Hvis vi skal følge en vanlig antagelse i kvantitativ tekstanalyse -- ``sekk med ord'' eller \emph{bag of words} -- skal vi kunne forstå innholdet i en tekst hvis vi deler opp teksten i segmenter, putter det i en pose, rister posen og tømmer det på et bord. Da vil denne sangen for eksempel se slik ut:

\begin{Shaded}
\begin{Highlighting}[]
\FunctionTok{library}\NormalTok{(stringr)}

\NormalTok{regndans }\OtherTok{\textless{}{-}} \FunctionTok{readLines}\NormalTok{(}\StringTok{"../data/regndans.txt"}\NormalTok{)}

\NormalTok{bow }\OtherTok{\textless{}{-}}\NormalTok{ regndans }\SpecialCharTok{\%\textgreater{}\%} \FunctionTok{str\_split}\NormalTok{(}\StringTok{"}\SpecialCharTok{\textbackslash{}\textbackslash{}}\StringTok{s"}\NormalTok{) }\SpecialCharTok{\%\textgreater{}\%} \FunctionTok{unlist}\NormalTok{()}

\FunctionTok{set.seed}\NormalTok{(}\DecValTok{984301}\NormalTok{)}

\NormalTok{bow[}\FunctionTok{sample}\NormalTok{(}\DecValTok{1}\SpecialCharTok{:}\FunctionTok{length}\NormalTok{(bow))]}
\end{Highlighting}
\end{Shaded}

\begin{verbatim}
##   [1] "begynner"           "kaffe"              "i"                 
##   [4] "på"                 "Ta"                 "backflip"          
##   [7] "Prøver"             "rustfarva,"         "når"               
##  [10] "Gresstrå"           "Drikke"             "skinnjakke"        
##  [13] "er"                 "I"                  "på"                
##  [16] "I"                  "TV-middager"        "av"                
##  [19] "Bare"               "Se"                 "med"               
##  [22] "krystalliserer"     "mеd"                "hele"              
##  [25] "Se"                 "Bjørkeblader"       "hele"              
##  [28] "i"                  "i"                  "hjem"              
##  [31] "i"                  "smilehulla"         "jeg"               
##  [34] "livet"              "Tusen"              "varmluftsballonger"
##  [37] "noen"               "dine"               "det"               
##  [40] "i"                  "[?]"                "nå,"               
##  [43] "opp"                "avgårde"            "bratwürst"         
##  [46] "det"                "endorfinene"        "Hårfestet"         
##  [49] "Gå"                 "Hasle"              "gule"              
##  [52] "høsten,"            "ass"                "Oslofjorden"       
##  [55] "gutt"               "og"                 "barnehager,"       
##  [58] "alt"                "og"                 "løsne"             
##  [61] "busskur"            "å"                  "året,"             
##  [64] "[?]"                "Også"               "til"               
##  [67] "Regndanse"          "T-banen"            "altså"             
##  [70] "hundre"             "livet"              "Hente"             
##  [73] "gråne"              "glass"              "blir"              
##  [76] "rekke"              "begynner"           "Våkne"             
##  [79] "dragepust"          "forbi"              "er"                
##  [82] "hagle"              "tar"                "å"                 
##  [85] "koppеr"             "i"                  "Løpe"              
##  [88] "på"                 "å"                  "Hage"              
##  [91] "Lage"               "si"                 "En"                
##  [94] "øl,"                "Ikke"               "og"                
##  [97] "en"                 "ass"                "flyet,"            
## [100] "sammen"             "nabolaget"          "trampoline"        
## [103] "ligge"              "Ringe"              "og"                
## [106] "kveld"              "i"                  "fly"               
## [109] "under"              "Nakenbade"          "går"               
## [112] "Grille"             "kveld"              "hos"               
## [115] "på"                 "seg"                "august"            
## [118] "Botanisk"
\end{verbatim}

De fleste (som ikke kan sangen fra før) vil ha vanskelig å forstå hva den egentlig handler om bare ved å se på dette. Vi kan identifisere meningsbærende ord som ``Oslofjorden'', ``Grille'', ``trampoline'', ``dragepust'', med mer. Likevel er det vanskelig å skjønne hva låtskriveren egentlig vil formidle med denne teksten. Det er dette som gjør ``sekk med ord''-antagelsen veldig streng. Språk er veldig komplekst og ordene i en tekst kan endre mening drastisk bare ved å se på en liten del av konteksten de dukker opp i. Om vi bare ser på linjen som inneholder orded ``dragepust'', innser vi fort at konteksten rundt ordet gir oss et veldig tydelig bilde av hva låtskriveren mener med akkurat den linjen:

\begin{Shaded}
\begin{Highlighting}[]
\NormalTok{regndans[}\FunctionTok{which}\NormalTok{(}\FunctionTok{str\_detect}\NormalTok{(regndans, }\StringTok{"dragepust"}\NormalTok{))]}
\end{Highlighting}
\end{Shaded}

\begin{verbatim}
## [1] "Våkne opp mеd dragepust"
\end{verbatim}

Likevel gir det oss ikke et godt bilde på hva teksten handler om i sin helhet. Det får vi bare sett ved å se på hele teksten:

\begin{verbatim}
## I kveld er nå, og året, alt av det
## Bare hele livet
## Løpe under busskur når det begynner å hagle
## Ikke rekke flyet, ligge sammen i Botanisk Hage
## Nakenbade i Oslofjorden
## Ringe på hos noen i nabolaget
## Lage TV-middager
## [?]
## Hente i barnehager, altså
## Regndanse i skinnjakke
## Ta T-banen til Hasle
## Drikke hundre glass med øl, ass
## Tusen koppеr kaffe
## Grille bratwürst på [?]
## Våkne opp mеd dragepust
## Se varmluftsballonger
## Bjørkeblader i august blir gule
## Også rustfarva, og løsne og fly avgårde
## Gresstrå på høsten, ass
## Hårfestet begynner å gråne
## Gå hjem og går forbi
## En gutt tar backflip på en trampoline
## Se endorfinene krystalliserer seg i smilehulla dine
## Prøver jeg å si
## I kveld er hele livet
\end{verbatim}

Nå teksten gir mening! Tolkninger kan selvfølgelig variere fra individ til individ og den ``riktige'' tolkningen, er det bare forfatteren som vet hva er. Personlig tolker jeg denne teksten som et utløp for frustrasjon under corona-pandemien, og prospektene ved livet når samfunnet gjenåpnes, fordi jeg hørte den for første gang under nedstengningen.

\hypertarget{fjerne-trekk}{%
\section{Fjerne trekk?}\label{fjerne-trekk}}

\hypertarget{punktsetting}{%
\subsection{Punktsetting}\label{punktsetting}}

\hypertarget{stoppord}{%
\subsection{Stoppord}\label{stoppord}}

\hypertarget{rotform-av-ord}{%
\section{Rotform av ord}\label{rotform-av-ord}}

\hypertarget{stemming}{%
\subsection{Stemming}\label{stemming}}

\hypertarget{lemmatisering}{%
\subsection{Lemmatisering}\label{lemmatisering}}

\hypertarget{ngrams}{%
\section{ngrams}\label{ngrams}}

\hypertarget{taledeler-parts-of-speech}{%
\section{Taledeler (parts of speech)}\label{taledeler-parts-of-speech}}

\hypertarget{sup}{%
\chapter{Veildedet læring}\label{sup}}

\hypertarget{unsup}{%
\chapter{Ikke-veiledet læring}\label{unsup}}

\hypertarget{ordboker}{%
\chapter{Ordbøker}\label{ordboker}}

\hypertarget{tekststats}{%
\chapter{Tekststatistikk}\label{tekststats}}

\hypertarget{likhet}{%
\section{Likhet}\label{likhet}}

\hypertarget{avstand}{%
\section{Avstand}\label{avstand}}

\hypertarget{lesbarhet}{%
\section{Lesbarhet}\label{lesbarhet}}

\hypertarget{uttrykk}{%
\section{Uttrykk}\label{uttrykk}}

\hypertarget{sentiment}{%
\chapter{Sentiment}\label{sentiment}}

\hypertarget{norsentlex}{%
\section{NorSentLex}\label{norsentlex}}

\hypertarget{topicmod}{%
\chapter{Temamodellering}\label{topicmod}}

\hypertarget{posisjon}{%
\chapter{Latente posisjoner}\label{posisjon}}

\hypertarget{oppsummering}{%
\chapter{Oppsummering}\label{oppsummering}}

  \bibliography{stv2022.bib,book.bib,packages.bib}

\end{document}
