% Options for packages loaded elsewhere
\PassOptionsToPackage{unicode}{hyperref}
\PassOptionsToPackage{hyphens}{url}
%
\documentclass[
]{book}
\usepackage{amsmath,amssymb}
\usepackage{lmodern}
\usepackage{iftex}
\ifPDFTeX
  \usepackage[T1]{fontenc}
  \usepackage[utf8]{inputenc}
  \usepackage{textcomp} % provide euro and other symbols
\else % if luatex or xetex
  \usepackage{unicode-math}
  \defaultfontfeatures{Scale=MatchLowercase}
  \defaultfontfeatures[\rmfamily]{Ligatures=TeX,Scale=1}
\fi
% Use upquote if available, for straight quotes in verbatim environments
\IfFileExists{upquote.sty}{\usepackage{upquote}}{}
\IfFileExists{microtype.sty}{% use microtype if available
  \usepackage[]{microtype}
  \UseMicrotypeSet[protrusion]{basicmath} % disable protrusion for tt fonts
}{}
\makeatletter
\@ifundefined{KOMAClassName}{% if non-KOMA class
  \IfFileExists{parskip.sty}{%
    \usepackage{parskip}
  }{% else
    \setlength{\parindent}{0pt}
    \setlength{\parskip}{6pt plus 2pt minus 1pt}}
}{% if KOMA class
  \KOMAoptions{parskip=half}}
\makeatother
\usepackage{xcolor}
\IfFileExists{xurl.sty}{\usepackage{xurl}}{} % add URL line breaks if available
\IfFileExists{bookmark.sty}{\usepackage{bookmark}}{\usepackage{hyperref}}
\hypersetup{
  pdftitle={STV2022 -- Store tekstdata},
  pdfauthor={Solveig Bjørkholt og Martin Søyland},
  hidelinks,
  pdfcreator={LaTeX via pandoc}}
\urlstyle{same} % disable monospaced font for URLs
\usepackage{color}
\usepackage{fancyvrb}
\newcommand{\VerbBar}{|}
\newcommand{\VERB}{\Verb[commandchars=\\\{\}]}
\DefineVerbatimEnvironment{Highlighting}{Verbatim}{commandchars=\\\{\}}
% Add ',fontsize=\small' for more characters per line
\usepackage{framed}
\definecolor{shadecolor}{RGB}{248,248,248}
\newenvironment{Shaded}{\begin{snugshade}}{\end{snugshade}}
\newcommand{\AlertTok}[1]{\textcolor[rgb]{0.94,0.16,0.16}{#1}}
\newcommand{\AnnotationTok}[1]{\textcolor[rgb]{0.56,0.35,0.01}{\textbf{\textit{#1}}}}
\newcommand{\AttributeTok}[1]{\textcolor[rgb]{0.77,0.63,0.00}{#1}}
\newcommand{\BaseNTok}[1]{\textcolor[rgb]{0.00,0.00,0.81}{#1}}
\newcommand{\BuiltInTok}[1]{#1}
\newcommand{\CharTok}[1]{\textcolor[rgb]{0.31,0.60,0.02}{#1}}
\newcommand{\CommentTok}[1]{\textcolor[rgb]{0.56,0.35,0.01}{\textit{#1}}}
\newcommand{\CommentVarTok}[1]{\textcolor[rgb]{0.56,0.35,0.01}{\textbf{\textit{#1}}}}
\newcommand{\ConstantTok}[1]{\textcolor[rgb]{0.00,0.00,0.00}{#1}}
\newcommand{\ControlFlowTok}[1]{\textcolor[rgb]{0.13,0.29,0.53}{\textbf{#1}}}
\newcommand{\DataTypeTok}[1]{\textcolor[rgb]{0.13,0.29,0.53}{#1}}
\newcommand{\DecValTok}[1]{\textcolor[rgb]{0.00,0.00,0.81}{#1}}
\newcommand{\DocumentationTok}[1]{\textcolor[rgb]{0.56,0.35,0.01}{\textbf{\textit{#1}}}}
\newcommand{\ErrorTok}[1]{\textcolor[rgb]{0.64,0.00,0.00}{\textbf{#1}}}
\newcommand{\ExtensionTok}[1]{#1}
\newcommand{\FloatTok}[1]{\textcolor[rgb]{0.00,0.00,0.81}{#1}}
\newcommand{\FunctionTok}[1]{\textcolor[rgb]{0.00,0.00,0.00}{#1}}
\newcommand{\ImportTok}[1]{#1}
\newcommand{\InformationTok}[1]{\textcolor[rgb]{0.56,0.35,0.01}{\textbf{\textit{#1}}}}
\newcommand{\KeywordTok}[1]{\textcolor[rgb]{0.13,0.29,0.53}{\textbf{#1}}}
\newcommand{\NormalTok}[1]{#1}
\newcommand{\OperatorTok}[1]{\textcolor[rgb]{0.81,0.36,0.00}{\textbf{#1}}}
\newcommand{\OtherTok}[1]{\textcolor[rgb]{0.56,0.35,0.01}{#1}}
\newcommand{\PreprocessorTok}[1]{\textcolor[rgb]{0.56,0.35,0.01}{\textit{#1}}}
\newcommand{\RegionMarkerTok}[1]{#1}
\newcommand{\SpecialCharTok}[1]{\textcolor[rgb]{0.00,0.00,0.00}{#1}}
\newcommand{\SpecialStringTok}[1]{\textcolor[rgb]{0.31,0.60,0.02}{#1}}
\newcommand{\StringTok}[1]{\textcolor[rgb]{0.31,0.60,0.02}{#1}}
\newcommand{\VariableTok}[1]{\textcolor[rgb]{0.00,0.00,0.00}{#1}}
\newcommand{\VerbatimStringTok}[1]{\textcolor[rgb]{0.31,0.60,0.02}{#1}}
\newcommand{\WarningTok}[1]{\textcolor[rgb]{0.56,0.35,0.01}{\textbf{\textit{#1}}}}
\usepackage{longtable,booktabs,array}
\usepackage{calc} % for calculating minipage widths
% Correct order of tables after \paragraph or \subparagraph
\usepackage{etoolbox}
\makeatletter
\patchcmd\longtable{\par}{\if@noskipsec\mbox{}\fi\par}{}{}
\makeatother
% Allow footnotes in longtable head/foot
\IfFileExists{footnotehyper.sty}{\usepackage{footnotehyper}}{\usepackage{footnote}}
\makesavenoteenv{longtable}
\usepackage{graphicx}
\makeatletter
\def\maxwidth{\ifdim\Gin@nat@width>\linewidth\linewidth\else\Gin@nat@width\fi}
\def\maxheight{\ifdim\Gin@nat@height>\textheight\textheight\else\Gin@nat@height\fi}
\makeatother
% Scale images if necessary, so that they will not overflow the page
% margins by default, and it is still possible to overwrite the defaults
% using explicit options in \includegraphics[width, height, ...]{}
\setkeys{Gin}{width=\maxwidth,height=\maxheight,keepaspectratio}
% Set default figure placement to htbp
\makeatletter
\def\fps@figure{htbp}
\makeatother
\setlength{\emergencystretch}{3em} % prevent overfull lines
\providecommand{\tightlist}{%
  \setlength{\itemsep}{0pt}\setlength{\parskip}{0pt}}
\setcounter{secnumdepth}{5}
\usepackage{booktabs}
\ifLuaTeX
  \usepackage{selnolig}  % disable illegal ligatures
\fi
\usepackage[]{natbib}
\bibliographystyle{plainnat}

\title{STV2022 -- Store tekstdata}
\author{Solveig Bjørkholt og Martin Søyland}
\date{2022-06-22}

\begin{document}
\maketitle

{
\setcounter{tocdepth}{1}
\tableofcontents
}
\hypertarget{introduksjon}{%
\chapter{Introduksjon}\label{introduksjon}}

Velkommen til STV2022 -- Store teksdata!

Dette er en arbeidsbok som går gjennom de forskjellige delene i kurset, med tilhørende R-kode. Meningen med arbeidsboken, er at den kan brukes som forslag til implementering av metoder i semesteroppgaven. Merk likevel at dette ikke er en fasit!

\hypertarget{om-kurset}{%
\chapter{Om kurset}\label{om-kurset}}

I kurset skal vi bli kjent med analyseprosessen av store tekstdata: Hvordan samler man effektivt og redelig store mengder politiske tekster? Hva må til for å gjøre slike tekster klare for analyse? Og hvordan kan vi analysere tekstene?

Politikere og politiske partier produserer store mengder tekst hver dag. Om det er gjennom debatter, taler på Stortinget, lovforslag fra regjeringen, høringer, offentlige utredninger med mer, er digitaliserte politiske tekster i det offentlige blitt mer tilgjengelig de siste tiårene. Dette har åpnet et mulighetsrom for tekstanalyse som ikke var mulig/veldig vanskelig og tidkrevende før.

Det kan ofte være vanskelig å finne mønster som kan svare på spørsmål og teorier vi har i statsvitenskap i disse store tekstsamlingene. Derfor kan vi se til metoder innenfor maskinlæring for å analysere store samlinger av tekst systematisk. Samtidig er ikke alltid digitaliserte politiske tekster tilrettelagt for å analysers direkte. I disse tilfellene er god strukturering av rådata viktig.

Gjennom å delta i dette kurset vil du lære å søke i store mengder dokumenter, oppsummere disse på meningsfulle måter og indentifisere riktige analysemetoder for å teste statsvitenskaplige teorier med store tekstdata. Kurset vil dekke samling av store volum tekst fra offentlige kilder, strukturering og klargjøring av tekst for analyse og kvantitative tekstanalysemetoder.

\hypertarget{oppbygging-av-arbeidsboken}{%
\chapter{Oppbygging av arbeidsboken}\label{oppbygging-av-arbeidsboken}}

Under vil vi gå gjennom undervisningsopplegget, som arbeidsboken er lagt opp etter. Delene av boken er strukturert som følgende:

\begin{enumerate}
\def\labelenumi{\arabic{enumi}.}
\tightlist
\item
  \protect\hyperlink{anskaff}{Anskaffelse av tekst}
\item
  \protect\hyperlink{lastetekst}{Laste inn eksisterende tekstkilder}
\item
  \protect\hyperlink{prepros}{Forbehandling av tekst (preprosessering)}
\item
  \protect\hyperlink{sup}{Veiledet læring (supervised)}
\item
  \protect\hyperlink{unsup}{Ikke-veiledet læring (unsupervised)}
\item
  \protect\hyperlink{ordboker}{Ordbøker}
\item
  \protect\hyperlink{tekststats}{Tekstsatistikk}
\item
  \protect\hyperlink{sentiment}{Sentiment}
\item
  \protect\hyperlink{topicmod}{Temamodellering}
\item
  \protect\hyperlink{posisjon}{Latente posisjoner i tekst}
\end{enumerate}

\hypertarget{anbefalte-forberedelser}{%
\chapter{Anbefalte forberedelser}\label{anbefalte-forberedelser}}

Siden kurset krever noe forkunnskap om R og generell metodisk kompetanse, anbefaler vi å se over følgende materiale før kurset starter:

\begin{itemize}
\tightlist
\item
  \href{https://shinyibv02.uio.no/connect/\#/apps/55/access}{Arbeidsbøker for R ved UiO}
\item
  \href{https://github.com/liserodland/STV1020}{R materiale for STV1020}
\end{itemize}

\hypertarget{undervisning}{%
\chapter{Undervisning}\label{undervisning}}

Undervisningen i STV2022 består av 10 forelesninger og 5 seminarer. Vi vil bruke forelesningene til å oppsummere hovedkonseptene i hver ukes tema, både metodisk og anvendt. Seminarene vil ha hovedfokus på teknisk gjennomføring av tekstanalyse i R. Hvert seminar vil være delt i to med én del der seminarleder går gjennom ekstempler på kodeimplementering og én del der studentene kan jobbe med semesteroppgaven. Det er også verdt å merke seg at mange av implementeringene i kurset krever en del prøving og feiling.

Merk at det etter hvert seminar skal leveres inn et utkast av oppgaven for temaet man har gått gjennom i seminaret. Disse delene må bestås for å få vurdert semesteroppgave.

\hypertarget{forelesninger}{%
\section{Forelesninger}\label{forelesninger}}

De ti forelesningene har følgende timeplan (høsten 2022):

\begin{longtable}[]{@{}
  >{\raggedright\arraybackslash}p{(\columnwidth - 10\tabcolsep) * \real{0.1207}}
  >{\raggedright\arraybackslash}p{(\columnwidth - 10\tabcolsep) * \real{0.1121}}
  >{\raggedright\arraybackslash}p{(\columnwidth - 10\tabcolsep) * \real{0.3103}}
  >{\raggedright\arraybackslash}p{(\columnwidth - 10\tabcolsep) * \real{0.1034}}
  >{\raggedright\arraybackslash}p{(\columnwidth - 10\tabcolsep) * \real{0.2414}}
  >{\raggedright\arraybackslash}p{(\columnwidth - 10\tabcolsep) * \real{0.1121}}@{}}
\toprule
\begin{minipage}[b]{\linewidth}\raggedright
Dato
\end{minipage} & \begin{minipage}[b]{\linewidth}\raggedright
Tid
\end{minipage} & \begin{minipage}[b]{\linewidth}\raggedright
Aktivitet
\end{minipage} & \begin{minipage}[b]{\linewidth}\raggedright
Sted
\end{minipage} & \begin{minipage}[b]{\linewidth}\raggedright
Foreleser
\end{minipage} & \begin{minipage}[b]{\linewidth}\raggedright
Ressurser/pensum
\end{minipage} \\
\midrule
\endhead
ti. 23. aug. & 10:15--12:00 & Introduksjon & ES, Aud. 5 & S. Bjørkholt og M. Søyland & \citet{Grimmer2022} kap. 1-2 og 22, \citet{Lucas2015}, \citet{Silge2017} kap. 1, \citet{Pang2008} kap. 1 \\
ti. 30. aug. & 10:15--12:00 & Anskaffelse og innlasting av tekst & ES, Aud. 5 & M. Søyland & \citet{Grimmer2022} kap. 3-4, \citet{Cooksey2014} kap. 1, \citet{Wickham2020}, \citet{Hoyland2019} \\
ti. 6. sep. & 10:15--12:00 & Forbehandling av tekst 1 & ES, Aud. 5 & M. Søyland & \citet{Grimmer2022} kap. 5, \citet{Silge2017} kap. 3, \citet{Joergensen2019}, \citet{Barnes2019}, \citet{Benoit2020} \\
ti. 13. sep. & 10:15--12:00 & Forbehandling av tekst 2 & ES, Aud. 5 & S. Bjørkholt & \citet{Grimmer2022} kap. 9, \citet{Silge2017} kap. 4, \citet{Denny2018} \\
ti. 20. sep. & 10:15--12:00 & Bruke API -- Case: Stortinget & ES, Aud. 5 & M. Søyland & \citet{datastortinget2022}, \citet{Soeyland2022}, \citet{Finseraas2021} \\
ti. 11. okt. & 10:15--12:00 & Veiledet og ikke-veiledet læring & ES, Aud. 5 & S. Bjørkholt & \citet{Grimmer2022} kap. 10 og 17, \citet{DOrazio2014}, \citet{Feldman2006a}, \citet{Feldman2006b} \citet{Muchlinski2016} \\
ti. 18. okt. & 10:15--12:00 & Ordbøker, tekstlikhet og sentiment & ES, Aud. 5 & S. Bjørkholt & \citet{Grimmer2022} kap. 7 og 16, \citet{Silge2017} kap. 2, \citet{Pang2008} kap. 3-4, \citet{Liu2015}, Liu2015a \\
ti. 25. okt. & 10:15--12:00 & Temamodellering & ES, Aud. 5 & M. Søyland & \citet{Grimmer2022} kap. 13, \citet{Blei2012}, \citet{Silge2017} kap. 6, \citet{Roberts2014} \\
ti. 1. nov. & 10:15--12:00 & Estimere latent posisjon fra tekst & ES, Aud. 5 & S. Bjørkholt & \citet{Laver2003}, \citet{Slapin2008}, \citet{Lowe2017}, \citet{Lauderdale2016}, \citet{Peterson2018} \\
ti. 15. nov. & 10:15--12:00 & Oppsummering & ES, Aud. 5 & S. Bjørkholt og M. Søyland & \citet{Grimmer2022} kap 28, \citet{Wilkerson2017} \\
\bottomrule
\end{longtable}

\hypertarget{seminarer}{%
\section{Seminarer}\label{seminarer}}

\begin{longtable}[]{@{}ll@{}}
\toprule
Uke & Aktivitet \\
\midrule
\endhead
36 & Seminar 1: Anskaffe tekst og lage dtm i R \\
38 & Seminar 2: Preprosessering av tekstdata i R \\
42 & Seminar 3: Veiledet og ikke-veiledet læring i R \\
44 & Seminar 4: Modelleringsmetoder i R \\
46 & Seminar 5: Fra tekst til funn, Q\&A og oppgavehjelp \\
\bottomrule
\end{longtable}

Seminarledere:

\begin{itemize}
\tightlist
\item
  Eli Sofie Baltzersen \href{mailto:elibal@student.sv.uio.no}{\nolinkurl{elibal@student.sv.uio.no}}
\item
  Eric Gabo Ekeberg Nilsen \href{mailto:e.g.e.nilsen@stv.uio.no}{\nolinkurl{e.g.e.nilsen@stv.uio.no}}
\end{itemize}

\hypertarget{pensum}{%
\section{Pensum}\label{pensum}}

Som med alle andre fag, er det sterkt anbefalt at man ser over pensum før forelesning og seminar. Likevel kan pensum i kurset til tider være noe teknisk og uhåndterbart. Det er ikke forventet å \emph{pugge} formler eller fult ut forstå de matematiske beregninger bak de forskjellige modelleringsmetodene (selv om det åpenbart kan gjøre stoffet lettere å forstå). Hovedfokuset vårt vil være på å forstå hvilke operasjoner man må gjøre for å gå fra tekst til funn, hvilke antagelser man gjør i prosessen og klare å velge de riktige modellene for spørsmålet man vil ha svar på.

Grunnboken i pensum er \citet{Grimmer2022}. Vi vil lene oss mye på denne over alle temaene vi gjennomgår. For R har vi valgt å gjøre materialet så standardisert som mulig ved å bruke \texttt{tidyverse} så langt det lar seg gjøre. Spesielt bruker vi \citet{Silge2017} for implementeringer via R-pakken \texttt{tidytext}.

Vi har også lagt inn noen bidrag som anvender metodene vi går gjennom i løpet av kurset, som \citet{Peterson2018}, \citet{Lauderdale2016}, \citet{Hoyland2019}, \citet{Finseraas2021}, for å synliggjøre nytten av metodene i anvendt forskning.

\hypertarget{lastetekst}{%
\chapter{Laste inn tekstdata}\label{lastetekst}}

Om å laste tekst!

\hypertarget{anskaff}{%
\chapter{Anskaffelse av tekst}\label{anskaff}}

\hypertarget{html-skraping}{%
\section{.html-skraping}\label{html-skraping}}

Dette kan være lite gøy

\hypertarget{xml-skraping}{%
\section{.xml-skraping}\label{xml-skraping}}

Dette er enklere enn html

\hypertarget{json-skraping}{%
\section{.json-skraping}\label{json-skraping}}

Dette liker jeg ikke. Veldig rar datastrukturering

\hypertarget{curl}{%
\section{Curl}\label{curl}}

Ja, må vi egentlig snakke om curl

\hypertarget{apier}{%
\section{APIer}\label{apier}}

Kanskje bruke Stortinget som eksempel.

\hypertarget{prepros}{%
\chapter{Preprosessering}\label{prepros}}

Preprosessering er ganske viktig.

\hypertarget{sekk-med-ord}{%
\section{Sekk med ord}\label{sekk-med-ord}}

\emph{Alle} tekster er unike!

Ta for eksempel spor 6 på No.4-albumet vi allerede har jobbet med -- \emph{Regndans i skinnjakke}. Hvis vi skal følge en vanlig antagelse i kvantitativ tekstanalyse -- ``sekk med ord'' eller \emph{bag of words} -- skal vi kunne forstå innholdet i en tekst hvis vi deler opp teksten i segmenter, putter det i en pose, rister posen og tømmer det på et bord. Da vil denne sangen for eksempel se slik ut:

\begin{Shaded}
\begin{Highlighting}[]
\NormalTok{regndans }\OtherTok{\textless{}{-}} \FunctionTok{readLines}\NormalTok{(}\StringTok{"./data/regndans.txt"}\NormalTok{)}

\NormalTok{bow }\OtherTok{\textless{}{-}}\NormalTok{ regndans }\SpecialCharTok{\%\textgreater{}\%}
    \FunctionTok{str\_split}\NormalTok{(}\StringTok{"}\SpecialCharTok{\textbackslash{}\textbackslash{}}\StringTok{s"}\NormalTok{) }\SpecialCharTok{\%\textgreater{}\%}
    \FunctionTok{unlist}\NormalTok{()}

\FunctionTok{set.seed}\NormalTok{(}\DecValTok{984301}\NormalTok{)}

\FunctionTok{cat}\NormalTok{(bow[}\FunctionTok{sample}\NormalTok{(}\DecValTok{1}\SpecialCharTok{:}\FunctionTok{length}\NormalTok{(bow))])}
\end{Highlighting}
\end{Shaded}

\begin{verbatim}
## begynner kaffe i på Ta backflip Prøver rustfarva, når Gresstrå Drikke skinnjakke er I på I TV-middager av Bare Se med krystalliserer mеd hele Se Bjørkeblader hele i i hjem i smilehulla jeg livet Tusen varmluftsballonger noen dine det i [?] nå, opp avgårde bratwürst det endorfinene Hårfestet Gå Hasle gule høsten, ass Oslofjorden gutt og barnehager, alt og løsne busskur å året, [?] Også til Regndanse T-banen altså hundre livet Hente gråne glass blir rekke begynner Våkne dragepust forbi er hagle tar å koppеr i Løpe på å Hage Lage si En øl, Ikke og en ass flyet, sammen nabolaget trampoline ligge Ringe og kveld i fly under Nakenbade går Grille kveld hos på seg august Botanisk
\end{verbatim}

De fleste (som ikke kan sangen fra før) vil ha vanskelig å forstå hva den egentlig handler om bare ved å se på dette. Vi kan identifisere meningsbærende ord som ``Oslofjorden'', ``Grille'', ``trampoline'', ``dragepust'', med mer. Likevel er det vanskelig å skjønne hva låtskriveren egentlig vil formidle med denne teksten. Det er dette som gjør ``sekk med ord''-antagelsen veldig streng. Språk er veldig komplekst og ordene i en tekst kan endre mening drastisk bare ved å se på en liten del av konteksten de dukker opp i. Om vi bare ser på linjen som inneholder orded ``dragepust'', innser vi fort at konteksten rundt ordet gir oss et veldig tydelig bilde av hva låtskriveren mener med akkurat den linjen:

\begin{Shaded}
\begin{Highlighting}[]
\NormalTok{regndans[}\FunctionTok{which}\NormalTok{(}\FunctionTok{str\_detect}\NormalTok{(regndans, }\StringTok{"dragepust"}\NormalTok{))]}
\end{Highlighting}
\end{Shaded}

\begin{verbatim}
## [1] "Våkne opp mеd dragepust"
\end{verbatim}

Likevel gir det oss ikke et godt bilde på hva teksten handler om i sin helhet. Det får vi bare sett ved å se på hele teksten:

\begin{verbatim}
## I kveld er nå, og året, alt av det
## Bare hele livet
## Løpe under busskur når det begynner å hagle
## Ikke rekke flyet, ligge sammen i Botanisk Hage
## Nakenbade i Oslofjorden
## Ringe på hos noen i nabolaget
## Lage TV-middager
## [?]
## Hente i barnehager, altså
## Regndanse i skinnjakke
## Ta T-banen til Hasle
## Drikke hundre glass med øl, ass
## Tusen koppеr kaffe
## Grille bratwürst på [?]
## Våkne opp mеd dragepust
## Se varmluftsballonger
## Bjørkeblader i august blir gule
## Også rustfarva, og løsne og fly avgårde
## Gresstrå på høsten, ass
## Hårfestet begynner å gråne
## Gå hjem og går forbi
## En gutt tar backflip på en trampoline
## Se endorfinene krystalliserer seg i smilehulla dine
## Prøver jeg å si
## I kveld er hele livet
\end{verbatim}

Nå teksten gir mening! Tolkninger kan selvfølgelig variere fra individ til individ og den ``riktige'' tolkningen, er det bare forfatteren som vet hva er. Personlig tolker jeg denne teksten som et utløp for frustrasjon under corona-pandemien, og prospektene ved livet når samfunnet gjenåpnes, fordi jeg hørte den for første gang under nedstengningen.

Hovedpoenget med å vise dette er at \emph{sekk med ord}-antagelsen er veldig streng og ofte veldig urealistisk. Tekster (og språk generelt) er ekstremt komplekst. Det kan variere mellom geografiske områder (nasjoner, dialekter, osv), aldersgrupper, arenaer (talestol, dialog, monolog, osv), og individuell stil. Oppi alt dette skal vi prøve å finne mønster som sier noe om likhet/ulikhet mellom tekster. Heldigvis har vi flere verktøy som kan hjelpe oss i å lette litt på \emph{sekk med ord}-antagelsen. Men antagelsen vil likevel alltid være der, i en eller annen form. La oss se litt på hvilke teknikker vi kan bruke for å gjøre modellering av tekst noe mer omgripelig¸ men aller først skal vi se litt på hvilke trekk som muligens ikke gir oss så mye informasjon om det vi er ute etter, eller støy, som vi ofte vil fjerne.

\hypertarget{fjerne-trekk}{%
\section{Fjerne trekk?}\label{fjerne-trekk}}

\hypertarget{punktsetting}{%
\subsection{Punktsetting}\label{punktsetting}}

\hypertarget{stoppord}{%
\subsection{Stoppord}\label{stoppord}}

\hypertarget{rotform-av-ord}{%
\section{Rotform av ord}\label{rotform-av-ord}}

\hypertarget{stemming}{%
\subsection{Stemming}\label{stemming}}

\hypertarget{lemmatisering}{%
\subsection{Lemmatisering}\label{lemmatisering}}

\hypertarget{ngrams}{%
\section{ngrams}\label{ngrams}}

\hypertarget{taledeler-parts-of-speech}{%
\section{Taledeler (parts of speech)}\label{taledeler-parts-of-speech}}

\hypertarget{sup}{%
\chapter{Veildedet læring}\label{sup}}

\hypertarget{unsup}{%
\chapter{Ikke-veiledet læring}\label{unsup}}

\hypertarget{ordboker}{%
\chapter{Ordbøker}\label{ordboker}}

\hypertarget{tekststats}{%
\chapter{Tekststatistikk}\label{tekststats}}

\hypertarget{likhet}{%
\section{Likhet}\label{likhet}}

\hypertarget{avstand}{%
\section{Avstand}\label{avstand}}

\hypertarget{lesbarhet}{%
\section{Lesbarhet}\label{lesbarhet}}

\hypertarget{uttrykk}{%
\section{Uttrykk}\label{uttrykk}}

\hypertarget{sentiment}{%
\chapter{Sentiment}\label{sentiment}}

\hypertarget{norsentlex}{%
\section{NorSentLex}\label{norsentlex}}

\hypertarget{topicmod}{%
\chapter{Temamodellering}\label{topicmod}}

\hypertarget{posisjon}{%
\chapter{Latente posisjoner}\label{posisjon}}

\hypertarget{oppsummering}{%
\chapter{Oppsummering}\label{oppsummering}}

  \bibliography{stv2022.bib}

\end{document}
